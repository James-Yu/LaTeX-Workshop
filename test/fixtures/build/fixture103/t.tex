\documentclass{article}
\usepackage[utf8]{inputenc}

\usepackage[acronym, toc]{glossaries}

\usepackage[backend=bibtex]{biblatex}
\addbibresource{references.bib}

\makeglossaries

\newglossaryentry{latex}
{
    name=latex,
    description={Is a mark up language specially suited for scientific documents}
}

\newglossaryentry{maths}
{
    name=mathematics,
    description={Mathematics is what mathematicians do}
}

\newglossaryentry{formula}
{
    name=formula,
    description={A mathematical expression}
}

\newacronym{gcd}{GCD}{Greatest Common Divisor}

\newacronym{lcm}{LCM}{Least Common Multiple}


\newcommand{\addDividingLine}{\vspace{1cm}\hrule\vspace{1cm}}

\begin{document}

\tableofcontents

\addDividingLine

\section{First Section}

The \Gls{latex} typesetting markup language is specially suitable for documents that include \gls{maths}. \Glspl{formula} are rendered properly an easily once one gets used to the commands.

\addDividingLine

\section{Second Section}

\vspace{5mm}

Given a set of numbers, there are elementary methods to compute its \acrlong{gcd}, which is abbreviated \acrshort{gcd}. This process is similar to that used for the \acrfull{lcm}.

\addDividingLine

\printglossary

\addDividingLine

\printglossary[type=\acronymtype]

\addDividingLine

\section{Third Section}

\vspace{5mm}

Let's cite! The Einstein's journal paper \cite{einstein} and the Dirac's book \cite{dirac} are physics related items.

\addDividingLine

\printbibliography

\end{document}
